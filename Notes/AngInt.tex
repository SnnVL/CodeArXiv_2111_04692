\documentclass[10pt,a4paper,a4wide]{article}
%\documentclass[aps,floatfix,pra,preprint]{revtex4}
\usepackage{graphicx,bm,color}
\usepackage{array}
\usepackage{amsmath,amsfonts,amssymb}
%\usepackage[english]{babel}
\usepackage[utf8]{inputenc}
\usepackage[T1]{fontenc}

\usepackage{chngcntr}
\counterwithin*{paragraph}{subsubsection}


\newcommand{\be}{\begin{equation}}
\newcommand{\ee}{\end{equation}}
\newcommand{\bmul}{\begin{multline}}
\newcommand{\emul}{\end{multline}}
\newcommand{\bea}{\begin{eqnarray}}
\newcommand{\eea}{\end{eqnarray}}
\newcommand{\rr}{\mathbf{r}}
\newcommand{\kk}{\mathbf{k}}
\newcommand{\qq}{\mathbf{q}}
\newcommand{\KK}{\mathbf{K}}
\newcommand{\QQ}{\mathbf{Q}}
\newcommand{\pp}{\mathbf{p}}
\newcommand{\lr}{\mathbf{l}}
\newcommand{\zero}{\mathbf{0}}
\newcommand{\bra}[1]{\langle #1|}
\newcommand{\ket}[1]{|#1\rangle}

\newcommand{\ii}{\mathrm{i}}
\newcommand{\dd}{\mathrm{d}}
\newcommand{\eee}{\mathrm{e}}

\newcommand{\meanv}[1]{\langle #1 \rangle}
\newcommand{\meanvl}[1]{\overline{#1}}
\newcommand{\meanvlr}[1]{\left\langle #1 \right\rangle}


\newcommand{\comm}[2]{\left[ #1,#2 \right]}
\newcommand{\Tr}[1]{\text{Tr} \bb{#1}}
\newcommand{\g}[1]{«~#1~»}


\newcommand{\bb}[1]{\left( #1 \right)}
\newcommand{\bbr}[1]{\left. #1 \right)}
\newcommand{\bbl}[1]{\left( #1 \right.}
\newcommand{\bbcro}[1]{\left[ #1 \right]}
\newcommand{\bbcror}[1]{\left. #1 \right]}
\newcommand{\bbcrol}[1]{\left[ #1 \right.}
\newcommand{\bbaco}[1]{\left\{ #1 \right\}}
\newcommand{\bbacol}[1]{\left\{ #1 \right.}
\newcommand{\bbacor}[1]{\left. #1 \right\}}
\newcommand{\bbv}[1]{\left\vert #1 \right\vert}
\newcommand{\eval}[1]{\left. #1 \right|}



\newcommand{\hatk}[1]{\hat{#1}_\kk}

\newcommand{\deriv}[1]{i \hbar \frac{d #1}{dt}}
\newcommand{\derivsec}[1]{- \hbar^2 \frac{d^2 #1}{dt^2}}
\newcommand{\derivmu}[1]{\frac{d #1}{d{\mu}}}
\newcommand{\derivmut}[1]{\frac{d #1}{d\tilde{\mu}}}
\newcommand{\derivn}[1]{\frac{d #1}{d\bar{N}}}
\newcommand{\gO}{\tilde{g_0}}
\newcommand{\Xthet}{\frac{X}{\Theta}}

\newcommand{\abs}[1]{\left|#1\right|}


\newcolumntype{R}[1]{>{\raggedright}p{#1}}
\newcolumntype{L}[1]{>{\raggedleft}p{#1}}
\newcolumntype{C}[1]{>{\centering}p{#1}}


\newcolumntype{R}[1]{>{\raggedright}p{#1}}
\newcolumntype{L}[1]{>{\raggedleft}p{#1}}
\newcolumntype{C}[1]{>{\centering}p{#1}}

\usepackage{chngcntr}
\setcounter{secnumdepth}{5}
\renewcommand{\thesection}{\Roman{section}}
\renewcommand{\thesubsection}{\Roman{section}.\arabic{subsection}}
\renewcommand{\thesubsubsection}{\Roman{section}.\arabic{subsection}.\arabic{subsubsection}}
\renewcommand{\theparagraph}{\textit{\alph{paragraph}}.}

\def\noir{\bf \color{black}}
\def\bleu{\bf \color{blue}}
\def\rouge{ \color{red}}
\def\vers{\bf\color{green}}
\def\violet{\bf\color{magenta}}

\usepackage{geometry}
\geometry{vmargin=3cm}
\geometry{hmargin=1.5cm}

\begin{document}
\title{Analytic calculation of the angular integral of the self-energy}
\author{Hadrien $\to$ Senne}

\maketitle

%\section{Variable $t$}
%
%In this note, I derive analytic formulas for the angular integrals appearing in the single-particle self-energies:
%\bea
%I_{\pm\pm}(\omega)&=&\int_{-1}^1 \frac{\dd u}{2}\bb{1\pm\frac{\xi_{\kk-\qq}}{\epsilon_{\kk-\qq}}} \frac{\Delta}{\omega+\epsilon_{\kk-\qq}}\\ 
%I_{+-}(\omega)&=&\int_{-1}^1 \frac{\dd u}{2}{\frac{\Delta}{\epsilon_{\kk-\qq}}} \frac{\Delta}{\omega+\epsilon_{\kk-\qq}} 
%\eea
%We work here in units of $|\omega|$, setting $\check k^2=k^2/2m|\omega|$, $\check q^2=q^2/2m|\omega|$ and
%\be
%\check{\xi}_{\kk-\qq}=\frac{{\xi}_{\kk-\qq}}{|\omega|}=\xi_0-2\check k \check q u \quad \text{with} \quad  \xi_0=\check k^2+\check q^2-\mu/|\omega|
%\ee
%I use a Euler substitution to rationalize the integrand, setting
%\be
%\check{\epsilon}_{\kk-\qq}=t-2\check k\check q u \quad \Longleftrightarrow \quad u=\frac{\epsilon_0^2-t}{4\check k\check q (\xi_0-t)} 
%\ee
%\be
%\dd u = \frac{t^2-2t\xi_0+\epsilon_0^2}{4\check k\check q (\xi_0-t)^2} \dd t
%\ee
%where I use $\epsilon_0=\sqrt{\xi_0^2+\check\Delta^2}$. Note in passing that this change of variable is monotonous (and increasing) since $t^2-2t\xi_0+\epsilon_0^2>0$.
%In the new variable $t$, the integrals become
%\bea
%I_{++}&=&\frac{\check \Delta^3}{2\check k\check q }\int_{t_{\rm min}}^{t_{\rm max}}  \frac{\dd t}{(t-\xi_0)P(t)}\\ 
%I_{--}&=&\frac{\check \Delta}{2\check k\check q }\int_{t_{\rm min}}^{t_{\rm max}} \dd t \frac{t-\xi_0}{P(t)} \\
%I_{+-}&=&\frac{\check \Delta^2}{2\check k\check q }\int_{t_{\rm min}}^{t_{\rm max}} \frac{\dd t}{P(t)}
%\eea
%with the new integration boundaries $t_{\rm min}=\check\epsilon_{k+q}-2\check k \check q$ and $t_{\rm max}=\check\epsilon_{k-q}+2\check k \check q$ and the polynomial
%\be
%P(t)=t^2-2t(\xi_0-\text{sg}\omega)+\epsilon_0^2-2\text{sg}\omega\xi_0
%\ee
%whose roots are
%\bea
%t_1 &=& \xi_0-\text{sg}\omega+\sqrt{1-\check\Delta^2} \\
%t_2 &=& \xi_0-\text{sg}\omega-\sqrt{1-\check\Delta^2} 
%\eea
%Effecting the partial fraction decomposition of the integrals, we get
%\bea
%I_{++}&=&\frac{\check \Delta}{4\check k\check q} (-I_1 -\text{sg}\omega I_2 +2I_3) \\ 
%I_{--}&=& \frac{\check \Delta}{4\check k\check q}  (I_1-\text{sg}\omega I_2)\\
%I_{+-}&=& \frac{\check \Delta^2}{4\check k\check q} I_2
%\eea
%with
%\be
%I_1\equiv \int_{t_{\rm min}}^{t_{\rm max}} \dd t \bb{\frac{1}{t-t_1}+\frac{1}{t-t_2}} =\text{ln} \left\vert\frac{(t_{\rm max}-t_1)(t_{\rm max}-t_2)}{(t_{\rm min}-t_1)(t_{\rm min}-t_2)}\right\vert-\ii\pi \bb{\Theta(t_{\rm max}-t_1)\Theta(t_1-t_{\rm min})+\Theta(t_{\rm max}-t_2)\Theta(t_2-t_{\rm min})}
%\ee
%\begin{multline}
%I_2\equiv \int_{t_{\rm min}}^{t_{\rm max}} \frac{\dd t}{\sqrt{1-\check\Delta^2}} \bb{\frac{1}{t-t_1}-\frac{1}{t-t_2}} \\ 
%=\begin{cases} \frac{1}{\sqrt{1-\check\Delta^2}} \text{ln} {\frac{(t_{\rm max}-t_1)(t_{\rm min}-t_2)}{(t_{\rm max}-t_2)(t_{\rm min}-t_1)}} \quad  \text{if} \quad \Delta>|\omega| \\
% \frac{1}{\sqrt{1-\check\Delta^2}} \bbcro{\text{ln} \left\vert{\frac{(t_{\rm max}-t_1)(t_{\rm min}-t_2)}{(t_{\rm max}-t_2)(t_{\rm min}-t_1)}}\right\vert -\ii\pi \bb{\Theta(t_{\rm max}-t_1)\Theta(t_1-t_{\rm min})-\Theta(t_{\rm max}-t_2)\Theta(t_2-t_{\rm min})}\quad  \text{if} \quad  \Delta<|\omega|}
%\end{cases}
%\end{multline}
%\be
%I_3\equiv \int_{t_{\rm min}}^{t_{\rm max}} \frac{\dd t}{t-\xi_0}=\text{ln}\frac{\epsilon_{k-q}-\xi_{k-q}}{\epsilon_{k+q}-\xi_{k+q}}
%\ee

%%%%%%%%%%%%%%%%%%%%%%%%%%%%%%%%%%%%%%

\section{Analytic calculation of the angular integral}

In this note, I derive analytic formulas for the angular integrals appearing in the single-particle self-energies:
\bea
I_{U^2|V^2}(\omega)&=&\int_{-1}^1 \frac{\dd u}{2}\bb{1\pm\frac{\xi_{\kk-\qq}}{\epsilon_{\kk-\qq}}} \frac{\Delta}{\omega+\epsilon_{\kk-\qq}}\\ 
I_{UV}(\omega)&=&\int_{-1}^1 \frac{\dd u}{2}{\frac{\Delta}{\epsilon_{\kk-\qq}}} \frac{\Delta}{\omega+\epsilon_{\kk-\qq}} 
\eea
We work here in units of $\Delta$, setting $\check k^2=k^2/2m\Delta$, $\check q^2=q^2/2m\Delta$ and
\be
\check{\xi}_{\kk-\qq}=\frac{{\xi}_{\kk-\qq}}{\Delta}=\xi_0-2\check k \check q u \quad \text{with} \quad  \xi_0=\check k^2+\check q^2-\mu/\Delta
\ee
We use a Euler substitution to rationalize the integrand, setting
\be
\check{\epsilon}_{\kk-\qq}-\check{\xi}_{\kk-\qq}=x \quad \Longleftrightarrow \quad u=\frac{x^2+2\xi_0 x-1}{4\check k\check q x} 
\ee
\be
\dd u = \frac{x^2+1}{4\check k\check q x^2} \dd x
\ee
Note in passing that this change of variable is monotonous (and increasing).
In the new variable $x$, the integrals become
\bea
I_{U^2}&=&\frac{1}{2\check k\check q }\int_{x_{\rm min}}^{x_{\rm max}}  \frac{\dd x}{xP(x)}\\ 
I_{V^2}&=&\frac{1}{2\check k\check q }\int_{t_{\rm min}}^{x_{\rm max}} \dd x \frac{x}{P(x)} \\
I_{UV}&=&\frac{1}{2\check k\check q  }\int_{x_{\rm min}}^{x_{\rm max}} \frac{\dd x}{P(x)}
\eea
with the new integration boundaries $x_{\rm min}=\check\epsilon_{k+q}-\check\xi_{k+q}$ and $x_{\rm max}=\check\epsilon_{k-q}-\check\xi_{k-q}$ and the polynomial
\be
P(x)=x^2+2\check\omega x+1
\ee
whose roots are
\bea
x_1 &=& -\check\omega+\sqrt{\check\omega^2-1} +\ii\text{sg}(\omega)0^+\\
x_2 &=& -\check\omega-\sqrt{\check\omega^2-1}  -\ii\text{sg}(\omega)0^+
\eea
Effecting the partial fraction decomposition of the integrals, we get
\bea
I_{U^2}&=&\frac{-I_1 -\check\omega I_2 +2I_3}{4\check k\check q}  \\ 
I_{V^2}&=& \frac{I_1-\check\omega I_2}{4\check k\check q}  \\
I_{UV}&=& \frac{I_2}{4\check k\check q } 
\eea
with
\begin{multline}
I_1\equiv \int_{x_{\rm min}}^{x_{\rm max}} \dd x \bb{\frac{1}{x-x_1}+\frac{1}{x-x_2}} =\text{ln} \left\vert\frac{(x_{\rm max}-x_1)(x_{\rm max}-x_2)}{(x_{\rm min}-x_1)(x_{\rm min}-x_2)}\right\vert\\+\ii\pi \text{sg}(\omega)\bb{\Theta(x_{\rm max}-x_1)\Theta(x_1-x_{\rm min})-\Theta(x_{\rm max}-x_2)\Theta(x_2-x_{\rm min})}
\end{multline}
\begin{multline}
I_2\equiv \int_{x_{\rm min}}^{x_{\rm max}} \frac{\dd x}{\sqrt{\check\omega^2-1}} \bb{\frac{1}{x-x_1}-\frac{1}{x-x_2}} \\ 
=\begin{cases} \frac{1}{\ii\sqrt{1-\check\omega^2}} \text{ln} {\frac{(x_{\rm max}-x_1)(x_{\rm min}-x_2)}{(x_{\rm max}-x_2)(x_{\rm min}-x_1)}}= \frac{2}{\sqrt{1-\check\omega^2}}\bb{\text{arg}(x_{\rm max}-x_1)-\text{arg}(x_{\rm min}-x_1)} \quad  \text{if} \quad \Delta>|\omega| \\
 \frac{1}{\sqrt{\check\omega^2-1}} \bbcro{\text{ln} \left\vert{\frac{(x_{\rm max}-x_1)(x_{\rm min}-x_2)}{(x_{\rm max}-x_2)(x_{\rm min}-x_1)}}\right\vert +\ii\pi \text{sg}(\omega) \bb{\Theta(x_{\rm max}-x_1)\Theta(x_1-x_{\rm min})+\Theta(x_{\rm max}-x_2)\Theta(x_2-x_{\rm min})}}\quad  \text{if} \quad  \Delta<|\omega|
\end{cases}
\end{multline}
\be
I_3\equiv\text{ln}\frac{x_{\rm max}}{x_{\rm min}}
\ee

\section{UV behavior of the angular integral}

At $\omega\gg q^2,k^2,|\mu|,\Delta$, one has
\bea
I_{U^2}&\sim& \frac{{\frac{1}{x_{\rm min}}-\frac{1}{x_{\rm max}}}}{4\check k \check q \check \omega}\\
I_{V^2}&\sim& \frac{{x_{\rm max}-x_{\rm min}}}{4\check k \check q \check \omega} \\
I_{UV}&\sim& \frac{\log\frac{x_{\rm max}}{x_{\rm min}}}{4\check k \check q \check \omega} 
\eea
and at $\omega\approx q^2\gg k^2,|\mu|,\Delta$, one has
\bea
I_{U^2}&\sim& \frac{2}{\check \omega+\check q^2}\\
I_{V^2}&\sim& \frac{1}{2(\check \omega+\check q^2)\check q^4} \\
I_{UV}&\sim& \frac{1}{(\check \omega+\check q^2)\check q^2} 
\eea
\section{UV behavior of the spectral density}

We recall the pair-fluctuation matrix in the cartesian basis (in Randeria and Senne’s notations):
\bea
N_{++}(z,\qq)&=&M_{11}(z,\qq)=-\frac{mV}{4\pi  a}+\sum_{\kk}\bbcro{\frac{U_+^2U_-^2}{z-\epsilon_+-\epsilon_-}-\frac{V_+^2 V_-^2}{z+\epsilon_+ + \epsilon_-}+\frac{1}{2k^2}} \\
N_{--}(z,\qq)&=&M_{22}(z,\qq)=-\frac{mV}{4\pi  a}+\sum_{\kk}\bbcro{\frac{V_+^2V_-^2}{z-\epsilon_+-\epsilon_-}-\frac{U_+^2 U_-^2}{z+\epsilon_+ + \epsilon_-}+\frac{1}{2k^2}} \\
N_{+-}(z,\qq)&=&M_{12}(z,\qq)=\sum_{\kk}\bbcro{\frac{U_+ U_- V_+ V_- }{z+\epsilon_+ + \epsilon_-}-\frac{U_+ U_- V_+ V_- }{z-\epsilon_+-\epsilon_-}} 
\eea
At large $\omega$, those integrals are dominated by the large-$k$ region. We thus expand at $k^2\gg1,|\mu|/\Delta$ (but a priori $q\approx k$):
\bea
U_{\kk}&=&1+O(k^{-4}) \\
V_{\kk}&=&\frac{1}{2k^2}+O(k^{-4}) \\
\epsilon_{\kk+\qq/2} + \epsilon_{\kk-\qq/2}&=&2k^2+q^2/2 +O(1)
\eea
Here and in the followings, we drop the $\check \ $ denoting the dimensionless quantities. Everywhere, we use  the units of $\Delta$ (for instance $\check a =k_\Delta a$), except for $\check M_{i,j}=(2\pi)^3\Delta M_{i,j}/k_\Delta^3$).
\bea
\text{Re}  M_{22}(\omega,\qq)&=&2\pi(2\omega+q^2) \int_0^{+\infty}\frac{\dd k}{4k^2+2\omega+q^2} -\frac{\pi^2}{ a}+O(q^{-1})=\pi^2\bbcro{\frac{\sqrt{2\omega+q^2}}{2}-\frac{1}{\check a}} \\
\text{Im}  M_{22}(\omega+\ii 0^+,\qq)&=& 2\pi\text{Im}\sum_{\kk} \frac{V_+^2V_-^2}{\omega-\epsilon_+-\epsilon_-+\ii 0^+} +O(q^{-9}) \\
 &=& -2\pi  \text{Im} \int_0^{+\infty}\frac{\dd k}{(4k^2+q^2)^3(k+k_0)(k-k_0-\ii 0^+)}\bbcro{\frac{4}{(2k-q^2)}+\frac{4}{(2k+q^2)}+\frac{1}{kq}\text{ln}\frac{(2k+q)^2}{(2k-q)^2}}  \notag \\
  &=& \frac{\pi^2}{q^7} m_{22}\bb{\frac{2\omega}{q^2}} \quad \text{with} \quad  m_{22}(\alpha)=-\frac{4\sqrt{\alpha-1}}{(\alpha-2)^2 \alpha^2}  -\frac{\text{ln}\bbcro{\frac{\sqrt{\alpha-1}+1}{\sqrt{\alpha-1}-1}}^2}{ \alpha^3} \notag \\
M_{11}(\omega+\ii 0^+,\qq)&=&-2\pi(2\omega-q^2) \int_0^{+\infty}\frac{\dd k}{4k^2+q^2-2\omega-\ii 0^+} -\frac{\pi^2}{ a} +O(q^{-1})=-\pi^2\bbcro{\ii\frac{\sqrt{2\omega-q^2}}{2}+\frac{1}{\check a}}\\
M_{12}(\omega+\ii 0^+,\qq)&=&\sum_\kk \bbcro{\frac{ V_+ V_- }{z+\epsilon_+ + \epsilon_-}-\frac{ V_+ V_- }{z-\epsilon_+-\epsilon_-}}  +O(q^5)\\
&=&4\pi \int_0^{+\infty} \dd k\bbcro{\frac{1}{4k^2+2\omega+q^2}-\frac{1}{2\omega-4k^2-q^2+\ii 0^+}} \int_{-1}^1 \dd u \bbcro{\frac{1}{(k^2+kq+q^2/4)(k^2-kq+q^2/4)}} \notag \\
&=&\frac{\pi^2}{q^3}m_{12}\bb{\frac{2\omega}{q^2}} \quad \text{with} \quad m_{12}(\alpha)=\frac{4}{\pi}\mathcal{P}\int_0^{+\infty}\frac{t\dd t\text{Argth}\bb{\frac{2t}{t^2+1}}}{(1+t)^2-\alpha^2}+\frac{\ii}{\alpha}\text{Argth}\bb{\frac{2\sqrt{\alpha-1}}{\alpha}} \notag
\eea
For the pair-propagator $\Gamma$, this leads to
\bea
\text{Re}\Gamma_{22}(\omega,\qq) &=& \frac{1}{\pi^2\bbcro{\frac{\sqrt{2\omega+q^2}}{2}-\frac{1}{\check a}}} +O(q^{-3})\\
\text{Im}\Gamma_{22}(\omega+\ii0^+,\qq) &=& \frac{1}{\pi^2q^9}\gamma_{22}\bb{{2\omega}/{q^2}} +O(q^{-10})\quad\text{with}\quad\gamma_{22}(\alpha)=-\frac{4}{\alpha+1}\bb{m_{22}(\alpha)-\frac{2\text{Re}m_{12}^2(\alpha)}{\sqrt{\alpha-1}}}\\
\Gamma_{11}(\omega+\ii 0^+,\qq) &=& -\frac{1}{\pi^2\bbcro{\ii\frac{\sqrt{2\omega-q^2}}{2}+\frac{1}{\check a}}} +O(q^{-2})\\
\Gamma_{12}(\omega+\ii 0^+,\qq) &=& \frac{1}{\pi^2 q^5} \frac{m_{12}(2\omega/{q^2})}{\sqrt{4\omega^2/{q^4}-1}} +O(q^{-6})\\
\eea

In the more stringent limiting case $z\gg q^2$ (even in the regime $q^2\approx 1,|\mu|$ {\color{red} (show this)}), those asymptotic behaviors simplify to
\bea
\text{Re} M_{22} &=& \frac{\pi^2\sqrt{\omega}}{\sqrt{2}}+O(1) \\
\text{Im} M_{22} &=& -\frac{\pi^2{\omega}^{7/2}}{\sqrt{2}}+O(\omega^4) \\
\text{Re} M_{11} &=& -\frac{\pi^2}{a}+O(\sqrt{\omega}) \\
\text{Im} M_{11} &=& \frac{\pi^2\sqrt{\omega}}{\sqrt{2}}+O(1) \\
M_{12} &=& \frac{\pi^2{\omega}^{3/2}}{\sqrt{2}}(\ii-1)+O(\omega^2) 
\eea

\bea
\text{Re} \Gamma_{22} &=& \frac{\sqrt{2}}{\pi^2\omega}+O({\omega}^{-1}) \\
\text{Im} \Gamma_{22} &=& \frac{\sqrt{2}}{\pi^2\omega^{9/2}}+O({\omega}^{-5}) \\
\text{Re} \Gamma_{11} &=&-\frac{2}{\pi^2 a\omega}+O(\omega^{-3/2}) \\
\text{Im} \Gamma_{11} &=&\frac{\sqrt{2}}{\pi^2\sqrt{\omega}}+ O(\omega^{-1}) \\
\Gamma_{12} &=& \frac{\sqrt{2}}{\pi^2\omega^{5/2}}(1+\ii)+O(\omega^{-3}) 
\eea

\section{UV behavior of the self-energy integrande}
At large $\omega$ but finite $q^2$, one has
\bea
-\rho_{++}(\omega) I_{V^2}(\omega\pm z_k)&\sim&\frac{\sqrt{2}\bb{x_{\rm max}-x_{\rm min}}}{4\pi^3 kq\omega^{3/2}}\\
-\rho_{--}(\omega) I_{U^2}(\omega\pm z_k)&\sim&\frac{\sqrt{2}\bb{x_{\rm min}^{-1}-x_{\rm max}^{-1}}}{4\pi^3 kq\omega^{11/2}}\\
-\rho_{+-}(\omega) I_{UV}(\omega\pm z_k)&\sim&\frac{\sqrt{2}\bb{\text{log}(x_{\rm max}/x_{\rm min})}}{4\pi^3 kq\omega^{7/2}} 
\eea
from which we get
\bea
J_{++}(q)&\equiv&-\int_{\omega_{0}}^{+\infty}\rho_{++}(\omega) I_{V^2}(\omega\pm z_k)=\frac{\sqrt{2}\bb{x_{\rm max}-x_{\rm min}}}{2\pi^3 kq\omega_0^{1/2}} \\
J_{--}(q)&\equiv&-\int_{\omega_{0}}^{+\infty}\rho_{--}(\omega) I_{U^2}(\omega\pm z_k)=\frac{\sqrt{2}\bb{x_{\rm min}^{-1}-x_{\rm max}^{-1}}}{18\pi^3 kq\omega_0^{9/2}} \\
J_{+-}(q)&\equiv&-\int_{\omega_{0}}^{+\infty}\rho_{+-}(\omega) I_{UV}(\omega\pm z_k)=\frac{\sqrt{2}\bb{\text{log}(x_{\rm max}/x_{\rm min})}}{10\pi^3 kq\omega_0^{5/2}} 
\eea

Then, at $\omega\approx q^2$, the integral over energies runs from $\omega_0=q^2/2$ ($\alpha=1$) to infinity. One has:
\bea
q^2J_{++}(q)&=&\frac{2}{\sqrt{3}\pi^3q^5} \quad \text{such that} \quad \int_{q_{\rm max}}^{+\infty} q^2J_{++}(q)=\frac{1}{2\sqrt{3}\pi^3 q_{\rm max}^4} \\
q^2J_{--}(q)&=&\frac{c_{--}}{q^9} \quad \text{with} \quad c_{--}=\int_1^{+\infty} \frac{4\gamma_{22}(\alpha)}{\pi^3(\alpha+2)}=\infty \quad \text{(divergence in } \alpha=2 ) \\
q^2J_{+-}(q)&=&\frac{c_{+-}}{q^7} \quad \text{with} \quad c_{+-}=\int_1^{+\infty} \frac{4\text{Im}\, m_{12}(\alpha)}{\pi^3(\alpha+2)\sqrt{\alpha^2-1}} \simeq 0.0432101
\eea
hence
\bea
%\int_{q_{\rm 0}}^{+\infty}q^2J_{++}(q)&=& \frac{2c_{--}}{7q_0^7} \\
\int_{q_{\rm max}}^{+\infty}q^2J_{+-}(q)&=& \frac{c_{+-}}{6q_{\rm max}^6}
\eea
\bibliographystyle{unsrt}
\bibliography{/Users/hkurkjian/Documents/biblio}

\end{document}